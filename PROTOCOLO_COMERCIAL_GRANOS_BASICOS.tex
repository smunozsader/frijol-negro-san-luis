\section{PROTOCOLO COMERCIAL PARA GRANOS
BÁSICOS}\label{protocolo-comercial-para-granos-buxe1sicos}

\subsection{Frijol Negro San Luis Potosí - Comercialización Mayorista
México}\label{frijol-negro-san-luis-potosuxed---comercializaciuxf3n-mayorista-muxe9xico}

\textbf{Basado en Mejores Prácticas del Sector Agroalimentario Mexicano}

\begin{center}\rule{0.5\linewidth}{0.5pt}\end{center}

\subsubsection{ÍNDICE}\label{uxedndice}

\begin{enumerate}
\def\labelenumi{\arabic{enumi}.}
\tightlist
\item
  \hyperref[1-marco-regulatorio-y-normativo]{Marco Regulatorio y
  Normativo}
\item
  \hyperref[2-benchmarking-sector-granos-buxe1sicos]{Benchmarking Sector
  Granos Básicos}
\item
  \hyperref[3-proceso-de-comercializaciuxf3n-mayorista]{Proceso de
  Comercialización Mayorista}
\item
  \hyperref[4-documentaciuxf3n-y-certificaciones]{Documentación y
  Certificaciones}
\item
  \hyperref[5-estructura-de-presentaciuxf3n-comercial]{Estructura de
  Presentación Comercial}
\item
  \hyperref[6-tuxe9rminos-contractuales-estuxe1ndar]{Términos
  Contractuales Estándar}
\item
  \hyperref[7-speech-de-aproximaciuxf3n-comercial]{Speech de
  Aproximación Comercial}
\end{enumerate}

\begin{center}\rule{0.5\linewidth}{0.5pt}\end{center}

\subsection{1. MARCO REGULATORIO Y
NORMATIVO}\label{marco-regulatorio-y-normativo}

\subsubsection{1.1 Normativas Mexicanas
Aplicables}\label{normativas-mexicanas-aplicables}

\textbf{A) Calidad y Clasificación:} - \textbf{NMX-FF-038-SCFI-2016:}
Productos alimenticios no industrializados para consumo humano -
cereales - frijol común (Phaseolus vulgaris L.) -
\textbf{NOM-247-SSA1-2008:} Productos y servicios. Cereales y sus
productos. Cereales, harinas de cereales, sémolas o semolinas -
\textbf{Ley Federal de Sanidad Vegetal:} Certificación fitosanitaria
obligatoria

\textbf{B) Comercialización y Transporte:} - \textbf{NOM-068-SCFI-2000:}
Productos alimenticios - frijoles - especificaciones y métodos de prueba
- \textbf{Ley de Desarrollo Rural Sustentable:} Marco para
comercialización de productos agrícolas - \textbf{Reglamento de Control
Sanitario de Productos y Servicios:} COFEPRIS

\textbf{C) Pesos y Medidas:} - \textbf{NOM-030-SCFI-2006:} Información
comercial - declaración de cantidad en la etiqueta - \textbf{Ley Federal
sobre Metrología y Normalización:} Verificación de básculas y medidas

\subsubsection{1.2 Organismos de Certificación
Reconocidos}\label{organismos-de-certificaciuxf3n-reconocidos}

\textbf{Laboratorios Acreditados por EMA (Entidad Mexicana de
Acreditación):} - SENASICA (Servicio Nacional de Sanidad, Inocuidad y
Calidad Agroalimentaria) - CIATEJ (Centro de Investigación y Asistencia
en Tecnología del Estado de Jalisco) - Laboratorios Certificados por
COFEPRIS - SGS México, Bureau Veritas México

\subsubsection{1.3 Aspectos Fiscales y
Tributarios}\label{aspectos-fiscales-y-tributarios}

\textbf{Régimen Fiscal Aplicable:} - \textbf{Actividad Primaria:}
Enajenación de productos agrícolas (Tasa 0\% IVA) -
\textbf{Comercialización:} 16\% IVA en reventa - \textbf{IEPS:} No
aplicable para productos básicos - \textbf{Retenciones:} 10.67\% ISR en
algunos casos (Ley del ISR Art. 25-A)

\begin{center}\rule{0.5\linewidth}{0.5pt}\end{center}

\subsection{2. BENCHMARKING SECTOR GRANOS
BÁSICOS}\label{benchmarking-sector-granos-buxe1sicos}

\subsubsection{2.1 Análisis de Competidores
Mayoristas}\label{anuxe1lisis-de-competidores-mayoristas}

\textbf{A) Comercializadoras Establecidas:}

{\def\LTcaptype{none} % do not increment counter
\begin{longtable}[]{@{}
  >{\raggedright\arraybackslash}p{(\linewidth - 6\tabcolsep) * \real{0.1765}}
  >{\raggedright\arraybackslash}p{(\linewidth - 6\tabcolsep) * \real{0.2157}}
  >{\raggedright\arraybackslash}p{(\linewidth - 6\tabcolsep) * \real{0.2941}}
  >{\raggedright\arraybackslash}p{(\linewidth - 6\tabcolsep) * \real{0.3137}}@{}}
\toprule\noalign{}
\begin{minipage}[b]{\linewidth}\raggedright
Empresa
\end{minipage} & \begin{minipage}[b]{\linewidth}\raggedright
Cobertura
\end{minipage} & \begin{minipage}[b]{\linewidth}\raggedright
Volumen Anual
\end{minipage} & \begin{minipage}[b]{\linewidth}\raggedright
Especialización
\end{minipage} \\
\midrule\noalign{}
\endhead
\bottomrule\noalign{}
\endlastfoot
\textbf{Grupo Minsa} & Nacional & 500,000+ ton & Maíz, sorgo, frijol \\
\textbf{Comercializadora Zamorana} & Centro-Occidente & 200,000 ton &
Granos básicos \\
\textbf{Distribuidora Conasupo} & Sureste & 150,000 ton & Productos
básicos \\
\textbf{Agro Mayoristas del Bajío} & Centro & 100,000 ton &
Leguminosas \\
\end{longtable}
}

\textbf{B) Modelos de Negocio Prevalentes:}

\textbf{1. Modelo Tradicional (70\% del mercado):} - Acopio directo de
productores - Almacenamiento en centros de distribución - Venta a
mayoristas regionales - Márgenes: 15-25\%

\textbf{2. Modelo de Agricultura por Contrato (20\%):} - Contratos
adelantados con productores - Especificaciones de calidad predefinidas -
Precios acordados antes de siembra - Márgenes: 8-15\%

\textbf{3. Modelo de Intermediación Especializada (10\%):} - Agentes
comisionistas (como nuestro caso) - Sin inversión en inventarios -
Enfoque en desarrollo comercial - Comisiones: 4-8\%

\subsubsection{2.2 Benchmarking de Precios y
Márgenes}\label{benchmarking-de-precios-y-muxe1rgenes}

\textbf{Estructura de Precios Mercado Nacional (Datos SIAP 2025):}

{\def\LTcaptype{none} % do not increment counter
\begin{longtable}[]{@{}
  >{\raggedright\arraybackslash}p{(\linewidth - 8\tabcolsep) * \real{0.1000}}
  >{\raggedright\arraybackslash}p{(\linewidth - 8\tabcolsep) * \real{0.2125}}
  >{\raggedright\arraybackslash}p{(\linewidth - 8\tabcolsep) * \real{0.2250}}
  >{\raggedright\arraybackslash}p{(\linewidth - 8\tabcolsep) * \real{0.2375}}
  >{\raggedright\arraybackslash}p{(\linewidth - 8\tabcolsep) * \real{0.2250}}@{}}
\toprule\noalign{}
\begin{minipage}[b]{\linewidth}\raggedright
Región
\end{minipage} & \begin{minipage}[b]{\linewidth}\raggedright
Precio Productor
\end{minipage} & \begin{minipage}[b]{\linewidth}\raggedright
Precio Mayorista
\end{minipage} & \begin{minipage}[b]{\linewidth}\raggedright
Precio Detallista
\end{minipage} & \begin{minipage}[b]{\linewidth}\raggedright
Margen Mayorista
\end{minipage} \\
\midrule\noalign{}
\endhead
\bottomrule\noalign{}
\endlastfoot
\textbf{Zacatecas} & \$16,500/ton & \$24,000/ton & \$35,000/ton &
45\% \\
\textbf{Nayarit} & \$17,000/ton & \$25,500/ton & \$37,000/ton & 50\% \\
\textbf{San Luis Potosí} & \$17,500/ton & \$26,000/ton & \$38,000/ton &
49\% \\
\textbf{Sinaloa} & \$16,000/ton & \$23,500/ton & \$34,000/ton & 47\% \\
\end{longtable}
}

\textbf{Análisis:} Nuestro precio objetivo de \$28,930/ton está 11\% por
encima del promedio de mercado, justificado por: - Calidad Grado Primera
certificada - Origen San Luis Potosí (premium reconocido) - Volumen
consolidado (1,000 toneladas) - Servicio comercial integral

\subsubsection{2.3 Mejores Prácticas del
Sector}\label{mejores-pruxe1cticas-del-sector}

\textbf{A) Gestión de Calidad:} - Análisis de laboratorio por lote
(100\% de empresas líderes) - Certificación NMX obligatoria -
Trazabilidad completa desde origen - Segregación por calidades (Extra,
Primera, Segunda)

\textbf{B) Logística y Distribución:} - Transporte especializado con
control de temperatura y humedad - Seguros de carga por 100\% del valor
- Almacenes certificados con control de plagas - Sistema de inventarios
PEPS (Primero en Entrar, Primero en Salir)

\textbf{C) Términos Comerciales Estándar:} - Pago: 30-45 días fecha de
factura - Descuentos: 2-3\% por pronto pago (15 días) - Garantías: Aval
bancario o seguro de crédito - Devoluciones: Máximo 2\% por calidad

\begin{center}\rule{0.5\linewidth}{0.5pt}\end{center}

\subsection{3. PROCESO DE COMERCIALIZACIÓN
MAYORISTA}\label{proceso-de-comercializaciuxf3n-mayorista}

\subsubsection{3.1 FASE I: PREPARACIÓN Y DOCUMENTACIÓN (7-10
días)}\label{fase-i-preparaciuxf3n-y-documentaciuxf3n-7-10-duxedas}

\paragraph{3.1.1 Investigación del Cliente
Objetivo}\label{investigaciuxf3n-del-cliente-objetivo}

\textbf{A) Intelligence Comercial:}

\begin{verbatim}
- Análisis de sitio web y presencia digital
- Investigación de estructura organizacional
- Identificación de decisores clave
- Análisis de cartera actual de proveedores
- Verificación de capacidad financiera (Buró de Crédito Comercial)
- Referencias de otros proveedores actuales
\end{verbatim}

\textbf{B) Segmentación de Compradores:}

\textbf{Mayoristas Tipo A (\textgreater1,000 ton/año):} - Empresas
consolidadas con 10+ años en mercado - Cobertura multi-estatal -
Sistemas de calidad certificados - Términos de pago: 30-45 días

\textbf{Mayoristas Tipo B (200-1,000 ton/año):} - Empresas regionales
establecidas - Cobertura estatal o regional - Procesos de calidad
básicos - Términos de pago: 15-30 días

\paragraph{3.1.2 Preparación
Documental}\label{preparaciuxf3n-documental}

\subparagraph{A) Documentos Técnicos
Obligatorios:}\label{a-documentos-tuxe9cnicos-obligatorios}

\begin{enumerate}
\def\labelenumi{\arabic{enumi}.}
\tightlist
\item
  \textbf{Certificado NMX-FF-038-SCFI-2016} (vigencia 12 meses)
\item
  \textbf{Certificado Fitosanitario SENASICA} (vigencia 30 días)\\
\item
  \textbf{Análisis de Laboratorio Completo} (máximo 15 días)
\item
  \textbf{Certificado de Fumigación} (si aplica)
\item
  \textbf{Dictamen de Verificación de Pesos y Medidas}
\end{enumerate}

\subparagraph{B) Documentos
Comerciales:}\label{b-documentos-comerciales}

\begin{enumerate}
\def\labelenumi{\arabic{enumi}.}
\tightlist
\item
  \textbf{Registro Federal de Contribuyentes (RFC)}
\item
  \textbf{Cédula de Identificación Fiscal}
\item
  \textbf{Estados Financieros Básicos} (últimos 2 años)
\item
  \textbf{Referencias Comerciales} (mínimo 3)
\item
  \textbf{Póliza de Seguro de Responsabilidad Civil}
\end{enumerate}

\subparagraph{C) Documentos de
Producto:}\label{c-documentos-de-producto}

\begin{enumerate}
\def\labelenumi{\arabic{enumi}.}
\tightlist
\item
  \textbf{Ficha Técnica (Fact Sheet)}
\item
  \textbf{Información Nutricional Certificada}
\item
  \textbf{Fotografías Profesionales del Producto}
\item
  \textbf{Especificaciones de Empaque y Etiquetado}
\end{enumerate}

\paragraph{3.1.3 Preparación de
Muestras}\label{preparaciuxf3n-de-muestras}

\textbf{Protocolo de Muestreo (NMX-FF-038):}

\begin{verbatim}
• Volumen de muestra: 10 kg mínimo
• Empaque: Costal de polipropileno nuevo
• Etiquetado: "MUESTRA - NO APTA PARA VENTA"
• Documentación: Análisis del lote específico
• Conservación: Ambiente seco, <18°C
• Vigencia: 30 días máximo
\end{verbatim}

\subsubsection{3.2 FASE II: CONTACTO INICIAL Y APROXIMACIÓN (3-5
días)}\label{fase-ii-contacto-inicial-y-aproximaciuxf3n-3-5-duxedas}

\paragraph{3.2.1 Estrategia de Contacto}\label{estrategia-de-contacto}

\textbf{Opción A - Aproximación Directa (Recomendada):}

\begin{verbatim}
Medio: Email profesional + llamada de seguimiento
Target: Gerente de Compras / Director Comercial
Timing: Martes-Jueves, 9:00-11:00 AM o 2:00-4:00 PM
Duración: Email conciso + llamada máximo 3 minutos
\end{verbatim}

\textbf{Opción B - Aproximación por Referencias:}

\begin{verbatim}
Medio: Introducción por terceros (proveedores, clientes comunes)
Ventaja: Mayor credibilidad inicial
Consideración: Reciprocidad comercial
\end{verbatim}

\paragraph{3.2.2 Template de Email
Inicial}\label{template-de-email-inicial}

\begin{verbatim}
ASUNTO: Propuesta Comercial - Frijol Negro San Luis Potosí Grado Primera NMX

Estimado [Nombre],

Mi nombre es Sergio Muñoz de Alba Medrano, represento a productores especializados 
en frijol negro de San Luis Potosí con certificación NMX-FF-038 Grado Primera.

Conocemos el liderazgo de [EMPRESA] en la distribución de granos básicos en el 
Sureste y consideramos que nuestro producto puede agregar valor a su cartera:

PROPUESTA INMEDIATA:
• Producto: Frijol Negro San Luis Potosí Grado Primera
• Disponibilidad: 1,000 toneladas inmediatas
• Certificación: NMX-FF-038-SCFI-2016 vigente
• Precio: Competitivo vs. mercado actual
• Condiciones: Flexibles según sus necesidades

VENTAJA COMPETITIVA:
• Calidad premium certificada consistente
• Volumen significativo para planificación anual
• Origen San Luis Potosí (reconocido por calidad superior)
• Servicio comercial personalizado

Solicito una cita de 45 minutos para presentar nuestra propuesta formal y 
entregar muestras para su evaluación técnica.

¿Cuándo sería conveniente para usted?

Atentamente,

Sergio Muñoz de Alba Medrano
Agente Comercial Especializado en Granos Básicos
Cel: [Número] | Email: [Email]
Ruta: San Luis Potosí → Mérida, Yucatán
\end{verbatim}

\paragraph{3.2.3 Script Telefónico de
Seguimiento}\label{script-telefuxf3nico-de-seguimiento}

\begin{verbatim}
"Buenos días, habla Sergio Muñoz de Alba. Le envié un email ayer sobre una 
propuesta de frijol negro San Luis Potosí. ¿Tuvo oportunidad de revisarlo?

[Pausa para respuesta]

Represento a productores con 1,000 toneladas de frijol negro Grado Primera 
certificado NMX. El precio es competitivo y la calidad es superior al promedio 
de mercado.

¿Podríamos agendar 45 minutos esta semana para que le presente la propuesta 
y le entregue muestras para evaluación?

[Confirmar fecha, hora y lugar]

Perfecto, le confirmo por email. Muchas gracias por su tiempo."
\end{verbatim}

\subsubsection{3.3 FASE III: PRESENTACIÓN COMERCIAL FORMAL (1
día)}\label{fase-iii-presentaciuxf3n-comercial-formal-1-duxeda}

\paragraph{3.3.1 Estructura de Presentación (45-60
minutos)}\label{estructura-de-presentaciuxf3n-45-60-minutos}

\textbf{MINUTOS 0-5: APERTURA}

\begin{verbatim}
• Agradecimiento por el tiempo
• Presentación personal y credenciales
• Agenda de la reunión
• Entrega de tarjeta de presentación
\end{verbatim}

\textbf{MINUTOS 5-15: ANÁLISIS DE MERCADO}

\begin{verbatim}
• Situación actual del mercado de frijol negro
• Precios y tendencias (datos SIAP/SNIIM)
• Oportunidades de crecimiento en Sureste
• Posicionamiento de calidad premium
\end{verbatim}

\textbf{MINUTOS 15-25: PRESENTACIÓN DEL PRODUCTO}

\begin{verbatim}
• Muestra física del producto
• Certificaciones y análisis de laboratorio
• Comparativo vs. competencia actual
• Ventajas del origen San Luis Potosí
\end{verbatim}

\textbf{MINUTOS 25-40: PROPUESTA COMERCIAL}

\begin{verbatim}
• Precio y términos comerciales
• Volúmenes disponibles
• Condiciones de entrega
• Estructura de descuentos
• Garantías de calidad
\end{verbatim}

\textbf{MINUTOS 40-55: ANÁLISIS CONJUNTO}

\begin{verbatim}
• Necesidades específicas del cliente
• Adaptaciones a sus requerimientos
• Términos de exclusividad (si aplica)
• Cronograma de implementación
\end{verbatim}

\textbf{MINUTOS 55-60: CIERRE Y PRÓXIMOS PASOS}

\begin{verbatim}
• Entrega formal de muestra y documentación
• Timeline de evaluación técnica
• Fecha de siguiente contacto
• Intercambio de datos de contacto directo
\end{verbatim}

\paragraph{3.3.2 Materiales de
Presentación}\label{materiales-de-presentaciuxf3n}

\textbf{A) Presentación PowerPoint (12 slides máximo):}

\textbf{Slide 1: Portada} - Logo/marca del producto - ``Frijol Negro San
Luis Potosí - Propuesta Comercial'' - Fecha y datos de contacto

\textbf{Slide 2: Agenda de la Presentación} - Análisis de mercado (5
min) - Nuestro producto (10 min) - Propuesta comercial (15 min) -
Análisis conjunto (15 min) - Próximos pasos (5 min)

\textbf{Slide 3: Mercado de Frijol Negro en México} - Producción
nacional: 1.2 millones de toneladas - Consumo per cápita: 9.5 kg/año -
Crecimiento Sureste: 8\% anual - Oportunidad de mercado premium

\textbf{Slide 4: Análisis Competitivo} - Precios actuales mayoristas:
\$23,500-26,000/ton - Calidades predominantes: Segunda y Primera -
Orígenes principales: Zacatecas (40\%), Nayarit (25\%), SLP (15\%) -
Oportunidad de diferenciación por calidad

\textbf{Slide 5: Nuestro Producto - Frijol Negro San Luis} -
Certificación NMX-FF-038 Grado Primera - Origen: San Luis Potosí
(terroir reconocido) - Características superiores vs.~estándar mercado -
Proceso de selección y clasificación

\textbf{Slide 6: Certificaciones y Calidad} - Análisis de laboratorio
completo - Cumplimiento NMX-FF-038 - Certificación fitosanitaria
SENASICA - Trazabilidad completa desde origen

\textbf{Slide 7: Ventaja Competitiva} - Calidad consistente Grado
Primera - Volumen significativo (1,000 toneladas) - Precio competitivo
vs.~calidad equivalente - Servicio comercial personalizado

\textbf{Slide 8: Propuesta de Valor para su Empresa} - Margen atractivo:
20\% sobre precio de compra - Producto diferenciado para sus clientes -
Reducción de riesgo por calidad garantizada - Soporte técnico y
comercial continuo

\textbf{Slide 9: Términos Comerciales} - Precio: \$28,930/tonelada EXW
San Luis Potosí (Ex-Works) - Flete SLP-Destino: A cargo del comprador -
Seguro de transporte: A cargo del comprador - Pago: 30 días fecha de
factura - Descuento pronto pago: 2\% (15 días) - Volumen mínimo: 40
toneladas (1 tractocamión)

\textbf{Slide 10: Condiciones Operativas} - Destino: Mérida, Yucatán
(1,500 km desde origen) - Entregas programadas según sus necesidades -
Empaque: Costales polipropileno 50 kg - Análisis de calidad por lote
certificado - Opciones: EXW San Luis Potosí o DDP Mérida

\textbf{Slide 11: Cronograma Propuesto} - Evaluación técnica: 7-10 días
- Negociación términos: 3-5 días - Primera entrega: 15-20 días
post-contrato - Entregas subsecuentes: Programación mensual

\textbf{Slide 12: Próximos Pasos} - Evaluación de muestra por su equipo
técnico - Presentación a comité de compras (si aplica) - Negociación de
términos específicos - Firma de contrato marco

\textbf{B) Documentos de Soporte:} - Ficha técnica completa (2 páginas)
- Propuesta comercial formal (1 página) - Referencias comerciales (1
página) - Análisis de laboratorio certificado

\subsubsection{3.4 FASE IV: EVALUACIÓN Y SEGUIMIENTO (7-14
días)}\label{fase-iv-evaluaciuxf3n-y-seguimiento-7-14-duxedas}

\paragraph{3.4.1 Proceso de Evaluación del
Cliente}\label{proceso-de-evaluaciuxf3n-del-cliente}

\textbf{Evaluación Técnica (5-7 días):}

\begin{verbatim}
• Análisis organoléptico (color, textura, aroma)
• Pruebas de cocción (tiempo, rendimiento)
• Análisis de laboratorio independiente (opcional)
• Comparación vs. proveedores actuales
\end{verbatim}

\textbf{Evaluación Comercial (3-5 días):}

\begin{verbatim}
• Análisis de precios vs. mercado
• Evaluación de términos y condiciones
• Verificación de capacidad de suministro
• Validación de referencias comerciales
\end{verbatim}

\paragraph{3.4.2 Cronograma de
Seguimiento}\label{cronograma-de-seguimiento}

\textbf{Día 3 post-presentación:}

\begin{verbatim}
Email: "Seguimiento evaluación muestra frijol negro San Luis"
- Consultar si requieren información adicional
- Ofrecer visita a instalaciones de origen
- Reafirmar términos de la propuesta
\end{verbatim}

\textbf{Día 7:}

\begin{verbatim}
Llamada telefónica:
- Consultar avance de evaluación técnica
- Solicitar feedback preliminar
- Preguntar sobre timeline de decisión
\end{verbatim}

\textbf{Día 10:}

\begin{verbatim}
Email formal:
- Solicitar resultados de evaluación
- Ofrecer ajustes menores a la propuesta
- Confirmar vigencia de términos comerciales
\end{verbatim}

\textbf{Día 14:}

\begin{verbatim}
Llamada de cierre:
- Solicitar decisión definitiva
- Si es positiva: programar negociación de contrato
- Si es negativa: solicitar razones específicas para futuros contactos
\end{verbatim}

\subsubsection{3.5 FASE V: NEGOCIACIÓN CONTRACTUAL (5-10
días)}\label{fase-v-negociaciuxf3n-contractual-5-10-duxedas}

\paragraph{3.5.1 Términos Base No
Negociables}\label{tuxe9rminos-base-no-negociables}

\textbf{Calidad:}

\begin{verbatim}
• Estándar mínimo: NMX-FF-038 Grado Primera
• Humedad máxima: 13%
• Impurezas máximas: 0.5%
• Granos dañados: <2%
• Certificación obligatoria por lote
\end{verbatim}

\textbf{Precio:}

\begin{verbatim}
• Precio base: $28,930/tonelada
• Mínimo aceptable: $28,500/tonelada
• Ajustes: Máximo 5% trimestral según SIAP
• Descuentos por volumen: Negociables >200 toneladas
\end{verbatim}

\paragraph{3.5.2 Términos Flexibles de
Negociación}\label{tuxe9rminos-flexibles-de-negociaciuxf3n}

\textbf{Comerciales:}

\begin{verbatim}
• Términos de pago: 15-45 días
• Descuento pronto pago: 1.5-2.5%
• Exclusividad territorial: Negociable
• Volúmenes de compromiso: Flexibles
\end{verbatim}

\textbf{Operativos:}

\begin{verbatim}
• Frecuencia de entregas: Semanal a mensual
• Tamaño de lote: 40-200 toneladas
• Empaque especial: Según requerimientos
• Almacenamiento temporal: Hasta 30 días
\end{verbatim}

\paragraph{3.5.3 Estructura del Contrato
Marco}\label{estructura-del-contrato-marco}

\textbf{CLÁUSULAS ESENCIALES:}

\textbf{1. Objeto del Contrato:}

\begin{verbatim}
Suministro de frijol negro (Phaseolus vulgaris L.) variedad negro, 
calidad Primera según NMX-FF-038-SCFI-2016, origen San Luis Potosí, 
en las cantidades, términos y condiciones establecidas en el presente contrato.
\end{verbatim}

\textbf{2. Especificaciones Técnicas:}

\begin{verbatim}
• Clasificación: NMX-FF-038 Grado Primera
• Humedad: Máximo 13% base húmeda
• Impurezas: Máximo 0.5% en peso
• Granos dañados: Máximo 2% en peso
• Empaque: Costales polipropileno 50±0.5 kg
• Marca: Según especificaciones del comprador
\end{verbatim}

\textbf{3. Precios y Términos Comerciales:}

\begin{verbatim}
• Precio base EXW: $28,930.00 MXN por tonelada
• Base EXW: Ex-Works almacén San Luis Potosí

• Precio base DDP: $31,097.00 MXN por tonelada
• Base DDP: Delivered Duty Paid almacén comprador en Mérida, Yucatán
• Moneda: Pesos mexicanos
• Ajustes: Trimestrales según SIAP (+/-5% máximo)
• Responsabilidades según INCOTERMS 2020
\end{verbatim}

\textbf{4. Condiciones de Pago:}

\begin{verbatim}
• Término: 30 días naturales fecha de factura
• Descuento: 2% por pago anticipado (15 días)
• Forma: Transferencia bancaria o cheque certificado
• Facturación: Contra entrega y recepción conforme
\end{verbatim}

\textbf{5. Entrega y Recepción:}

\begin{verbatim}
• Lugar: Almacén del comprador en Mérida, Yucatán
• Horario: Lunes a viernes 8:00-17:00 hrs
• Documentos: Remisión, certificados, análisis de calidad
• Verificación: Peso, calidad, documentación
\end{verbatim}

\textbf{6. Garantías de Calidad:}

\begin{verbatim}
• Análisis por lote certificado por laboratorio acreditado EMA
• Tolerancia de rechazo: <2% por motivos de calidad
• Reposición: Sin costo por defectos de calidad comprobados
• Seguros: Cobertura durante transporte
\end{verbatim}

\subsubsection{3.6 FASE VI: EJECUCIÓN Y SEGUIMIENTO (30+
días)}\label{fase-vi-ejecuciuxf3n-y-seguimiento-30-duxedas}

\paragraph{3.6.1 Preparación de Primera
Entrega}\label{preparaciuxf3n-de-primera-entrega}

\textbf{15 días antes:}

\begin{verbatim}
- Confirmación de programación con cliente
- Reserva de transporte especializado
- Análisis final de laboratorio del lote
- Preparación de documentación de embarque
- Coordinación con almacén de destino
\end{verbatim}

\textbf{7 días antes:}

\begin{verbatim}
- Confirmación de fecha exacta de entrega
- Envío de pre-aviso con datos del embarque
- Verificación de documentos de transporte
- Confirmación de personal de recepción
\end{verbatim}

\textbf{Día de entrega:}

\begin{verbatim}
- Acompañamiento personal de la entrega
- Supervisión de descarga y verificación
- Obtención de acuse de recibo firmado
- Documentación fotográfica del proceso
\end{verbatim}

\paragraph{3.6.2 Programa de Seguimiento
Post-Entrega}\label{programa-de-seguimiento-post-entrega}

\textbf{Semana 1:}

\begin{verbatim}
• Llamada de verificación de satisfacción
• Consulta sobre proceso de distribución
• Identificación de áreas de mejora
• Programación de segunda entrega
\end{verbatim}

\textbf{Semana 2:}

\begin{verbatim}
• Email de seguimiento comercial
• Solicitud de feedback de clientes finales
• Evaluación de rotación de inventario
• Ajustes operativos necesarios
\end{verbatim}

\textbf{Mes 1:}

\begin{verbatim}
• Reunión de evaluación mensual
• Análisis de KPIs comerciales
• Planificación de entregas futuras
• Identificación de oportunidades adicionales
\end{verbatim}

\begin{center}\rule{0.5\linewidth}{0.5pt}\end{center}

\subsection{4. DOCUMENTACIÓN Y
CERTIFICACIONES}\label{documentaciuxf3n-y-certificaciones}

\subsubsection{4.1 Certificados
Obligatorios}\label{certificados-obligatorios}

\paragraph{4.1.1 Certificado de Calidad
NMX-FF-038-SCFI-2016}\label{certificado-de-calidad-nmx-ff-038-scfi-2016}

\textbf{Parámetros de Evaluación:}

\begin{verbatim}
GRADO PRIMERA (Nuestro estándar):
• Humedad: Máximo 14%
• Granos sanos: Mínimo 97%
• Impurezas: Máximo 1%
• Granos quebrados: Máximo 4%
• Granos arrugados: Máximo 3%
• Granos manchados: Máximo 2%
• Materias extrañas: Máximo 0.25%
\end{verbatim}

\textbf{Metodología de Prueba (NMX-FF-038):}

\begin{verbatim}
1. Muestreo: 10% del lote, mínimo 2 kg
2. Determinación de humedad: Método de estufa 130°C±2°C
3. Separación manual: Granos sanos vs. defectuosos
4. Pesaje individual: Balanza analítica ±0.01g
5. Cálculo de porcentajes: Base peso total muestra
6. Dictamen: Cumple/No cumple especificación
\end{verbatim}

\paragraph{4.1.2 Certificado Fitosanitario
SENASICA}\label{certificado-fitosanitario-senasica}

\textbf{Requisitos de Emisión:}

\begin{verbatim}
• Inspección física del lote por inspector autorizado
• Ausencia de plagas cuarentenarias
• Tratamiento fitosanitario (si aplica)
• Muestreo para análisis de residuos de plaguicidas
• Vigencia: 30 días calendario
• Costo: $2,500-4,000 MXN según volumen
\end{verbatim}

\textbf{Plagas de Importancia Cuarentenaria:}

\begin{verbatim}
• Acanthoscelides obtectus (Gorgojo del frijol)
• Zabrotes subfasciatus (Gorgojo pinto del frijol)
• Callosobruchus maculatus (Gorgojo del caupí)
• Spodoptera frugiperda (Gusano cogollero)
\end{verbatim}

\subsubsection{4.2 Documentación
Comercial}\label{documentaciuxf3n-comercial}

\paragraph{4.2.1 Ficha Técnica Completa (Fact
Sheet)}\label{ficha-tuxe9cnica-completa-fact-sheet}

\begin{verbatim}
FRIJOL NEGRO SAN LUIS POTOSÍ - GRADO PRIMERA NMX-FF-038

INFORMACIÓN GENERAL:
• Nombre científico: Phaseolus vulgaris L.
• Variedad: Negro San Luis
• Origen: San Luis Potosí, México
• Clasificación: NMX-FF-038 Grado Primera
• Temporada: Primavera-Verano 2025

CARACTERÍSTICAS FÍSICAS:
• Forma: Arriñonada típica
• Color: Negro brillante uniforme
• Tamaño: Mediano (25-30 granos/10g)
• Textura: Lisa, sin rugosidades

ANÁLISIS BROMATOLÓGICO:
• Humedad: 12.5% ±0.5%
• Proteína: 22.3% base seca
• Carbohidratos: 62.1%
• Fibra cruda: 15.8%
• Cenizas: 4.2%
• Grasa: 1.6%

VALOR NUTRICIONAL (por 100g):
• Energía: 341 kcal
• Proteínas: 21.4 g
• Carbohidratos: 62.1 g
• Fibra dietética: 15.5 g
• Calcio: 160 mg
• Hierro: 5.9 mg
• Magnesio: 140 mg
• Potasio: 1,393 mg

CARACTERÍSTICAS CULINARIAS:
• Tiempo de cocción: 90-120 minutos
• Rendimiento de cocción: 2.3:1
• Capacidad de absorción: 130% peso seco
• Textura post-cocción: Firme, no desintegrable

MICROBIOLOGÍA (Límites máximos):
• Coliformes totales: <1,000 UFC/g
• E. coli: Ausencia en 25g
• Salmonella spp.: Ausencia en 25g
• Mohos y levaduras: <10,000 UFC/g

EMPAQUE Y ALMACENAMIENTO:
• Presentación: Costales polipropileno 50±0.5 kg
• Identificación: Etiqueta con lote, fecha, origen
• Almacenamiento: Lugar seco, ventilado, <18°C
• Vida útil: 24 meses en condiciones adecuadas
• Tratamiento: Fumigación con fosfuro de aluminio (si requerido)

CERTIFICACIONES:
• NMX-FF-038-SCFI-2016 Grado Primera
• Certificado Fitosanitario SENASICA vigente
• Análisis microbiológico laboratorio acreditado EMA
• Certificado de origen Gobierno de San Luis Potosí

TRAZABILIDAD:
• Productor: [Nombre y ubicación del predio]
• Fecha de cosecha: [Mes/Año]
• Lote de producción: [Código único]
• Fecha de procesamiento: [Fecha de selección]
• Responsable técnico: [Nombre del ingeniero]

CONTACTO:
Sergio Muñoz de Alba Medrano
Agente Comercial Especializado
Cel: [Número] | Email: [Email]
San Luis Potosí, México
\end{verbatim}

\begin{center}\rule{0.5\linewidth}{0.5pt}\end{center}

\subsection{5. ESTRUCTURA DE PRESENTACIÓN
COMERCIAL}\label{estructura-de-presentaciuxf3n-comercial}

\subsubsection{5.1 Deck de Presentación
Profesional}\label{deck-de-presentaciuxf3n-profesional}

\paragraph{5.1.1 Slide de Apertura (Impacto
Visual)}\label{slide-de-apertura-impacto-visual}

\begin{verbatim}
[IMAGEN: Granos de frijol negro San Luis en primer plano, calidad premium]

FRIJOL NEGRO SAN LUIS POTOSÍ
GRADO PRIMERA NMX-FF-038

"La calidad que su mercado demanda"

Propuesta Comercial para [NOMBRE EMPRESA]
Diciembre 2025

Sergio Muñoz de Alba Medrano
Especialista en Comercialización de Granos Básicos
\end{verbatim}

\paragraph{5.1.2 Análisis de Oportunidad de
Mercado}\label{anuxe1lisis-de-oportunidad-de-mercado}

\begin{verbatim}
EL MERCADO DE FRIJOL NEGRO EN SURESTE MÉXICO

DATOS CLAVE 2025:
• Consumo regional: 180,000 toneladas/año
• Crecimiento anual: 8.5%
• Precio promedio mayorista: $25,500/tonelada
• Calidad predominante: Segunda (65%) y Primera (35%)

OPORTUNIDAD IDENTIFICADA:
• Demanda insatisfecha de Grado Primera: 15,000 ton
• Premium por calidad superior: 15-20%
• Mercado dispuesto a pagar por diferenciación
• Consolidación de proveedores premium

VENTANA DE OPORTUNIDAD:
• Temporada alta: Enero-Abril 2026
• Inventarios bajos post-fiestas decembrinas
• Precios al alza según proyecciones SIAP
• Nuevas regulaciones de calidad en retail
\end{verbatim}

\paragraph{5.1.3 Propuesta de Valor
Diferenciada}\label{propuesta-de-valor-diferenciada}

\begin{verbatim}
POR QUÉ FRIJOL NEGRO SAN LUIS POTOSÍ

VENTAJAS COMPROBADAS:
- Origen Premium: San Luis Potosí = Calidad reconocida nacionalmente
- Terroir Único: Condiciones climáticas ideales para frijol negro
✓ Genética Superior: Variedades mejoradas con 25% más proteína
✓ Proceso Artesanal: Selección manual + tecnología de clasificación

DIFERENCIADORES VS. COMPETENCIA:
• Humedad controlada: 12.5% vs. 15% mercado estándar
• Uniformidad: >95% vs. 85% promedio sector
• Tiempo cocción: 90 min vs. 120 min competencia
• Rendimiento: 2.3:1 vs. 2:1 frijoles convencionales

CERTIFICACIONES ÚNICAS:
• NMX-FF-038 Grado Primera (solo 20% del mercado)
• Trazabilidad completa desde semilla
• Análisis nutricional certificado
• Proceso libre de químicos post-cosecha
\end{verbatim}

\subsubsection{5.2 Herramientas de
Demostración}\label{herramientas-de-demostraciuxf3n}

\paragraph{5.2.1 Kit de Muestras
Profesional}\label{kit-de-muestras-profesional}

\begin{verbatim}
CONTENIDO DEL KIT:
1. Muestra principal: 2 kg frijol negro San Luis
2. Muestra comparativa: 500g competencia directa
3. Análisis de laboratorio certificado
4. Ficha técnica completa
5. Instructivo de prueba de cocción
6. Formulario de evaluación sensorial

EMPAQUE:
• Caja de cartón corrugado con logo
• Bolsas herméticas individuales identificadas
• Etiquetas con códigos de lote y fecha
• Manual de evaluación incluido

DOCUMENTACIÓN:
• Certificado NMX del lote específico
• Análisis microbiológico vigente
• Certificado fitosanitario copia
• Propuesta comercial impresa
\end{verbatim}

\paragraph{5.2.2 Protocolo de Prueba
Técnica}\label{protocolo-de-prueba-tuxe9cnica}

\begin{verbatim}
EVALUACIÓN SUGERIDA PARA EL CLIENTE:

ANÁLISIS VISUAL (10 minutos):
1. Uniformidad de color y tamaño
2. Ausencia de granos partidos o dañados
3. Limpieza (ausencia de impurezas)
4. Brillo y apariencia general

PRUEBA DE COCCIÓN (2 horas):
1. Remojo: 8 horas en agua potable
2. Cocción: Agua hirviendo, sin sal
3. Tiempo hasta suavidad: Cronometrar
4. Rendimiento: Pesar antes y después
5. Textura final: Evaluación sensorial

EVALUACIÓN NUTRICIONAL:
1. Comparar análisis bromatológico
2. Contenido proteico vs. competencia
3. Fibra dietética y minerales
4. Digestibilidad y valor biológico

REPORTE DE RESULTADOS:
• Formato incluido en el kit
• Envío por email en 7 días
• Reunión de seguimiento programada
\end{verbatim}

\begin{center}\rule{0.5\linewidth}{0.5pt}\end{center}

\subsection{6. TÉRMINOS CONTRACTUALES
ESTÁNDAR}\label{tuxe9rminos-contractuales-estuxe1ndar}

\subsubsection{6.1 Condiciones Comerciales
Base}\label{condiciones-comerciales-base}

\paragraph{6.1.1 Estructura de Precios}\label{estructura-de-precios}

\begin{verbatim}
PRECIO BASE: $28,930.00 MXN por tonelada

BASE DE COTIZACIÓN: EXW (Ex-Works) San Luis Potosí
• Incluye: Producto, análisis de calidad, documentación, carga en almacén origen
• Excluye: Transporte, seguros, maniobras de descarga en destino
• Responsabilidad del comprador: Flete, seguros, gestión logística completa

OPCIÓN ALTERNATIVA: DDP (Delivered Duty Paid) Almacén Comprador
• Precio: $31,097/tonelada (incluye flete $2,167/ton)
• Incluye: Producto + transporte + seguros hasta almacén destino
• Responsabilidad vendedor: Entrega en almacén del comprador

DESCUENTOS POR VOLUMEN:
• 100-199 toneladas: Precio base
• 200-499 toneladas: 2% descuento ($28,350/ton)
• 500+ toneladas: 3% descuento ($28,062/ton)

AJUSTES DE PRECIO:
• Revisión: Trimestral
• Base: Índice SIAP + inflación
• Límite: ±5% por período
• Notificación: 30 días anticipados
\end{verbatim}

\paragraph{6.1.2 Términos de Pago}\label{tuxe9rminos-de-pago}

\begin{verbatim}
MODALIDAD ESTÁNDAR:
• Término: 30 días naturales fecha de factura
• Forma: Transferencia bancaria SPEI
• Descuento pronto pago: 2% (pago a 15 días)
• Intereses moratorios: 2% mensual (TIIE+10 puntos)

GARANTÍAS REQUERIDAS:
Primera transacción: Carta de crédito o aval bancario
Relación establecida (>6 meses): Sin garantías adicionales
Límite de crédito: Evaluación Bureau de Crédito Comercial

FACTURACIÓN:
• Régimen: Actividad empresarial
• CFDI: Clave 01010101 (Frijol)
• IVA: 0% (Actividad agropecuaria)
• Complemento: Carta Porte (transporte)
\end{verbatim}

\subsubsection{6.2 INCOTERMS 2020 - Definición de
Responsabilidades}\label{incoterms-2020---definiciuxf3n-de-responsabilidades}

\paragraph{6.2.1 Términos de Entrega
Disponibles}\label{tuxe9rminos-de-entrega-disponibles}

\textbf{OPCIÓN A: EXW (Ex-Works) San Luis Potosí}

\begin{verbatim}
PRECIO: $28,930/tonelada
RESPONSABILIDADES DEL VENDEDOR:
• Poner la mercancía a disposición en almacén San Luis Potosí
• Embalaje adecuado para transporte
• Documentación de exportación (certificados, facturas)
• Facilitar carga en vehículo del comprador

RESPONSABILIDADES DEL COMPRADOR:
• Contratar y pagar transporte completo
• Contratar y pagar seguros de mercancía
• Gestionar permisos y documentos de tránsito
• Asumir todos los riesgos desde almacén origen
• Descargar mercancía en destino
\end{verbatim}

\textbf{OPCIÓN B: DDP (Delivered Duty Paid) Almacén Comprador}

\begin{verbatim}
PRECIO: $31,097/tonelada
RESPONSABILIDADES DEL VENDEDOR:
• Entregar mercancía en almacén del comprador
• Contratar y pagar transporte completo
• Contratar y pagar seguros de mercancía
• Gestionar todos los permisos y documentos
• Asumir riesgos hasta entrega en destino

RESPONSABILIDADES DEL COMPRADOR:
• Facilitar acceso para descarga
• Verificar conformidad de mercancía
• Pagar según términos acordados
\end{verbatim}

\paragraph{6.2.2 Condiciones
Logísticas}\label{condiciones-loguxedsticas}

\begin{verbatim}
PROGRAMACIÓN DE ENTREGAS:
• Anticipación mínima: 7 días hábiles
• Volumen mínimo: 40 toneladas (1 tractocamión)
• Volumen máximo: 200 toneladas por entrega
• Frecuencia sugerida: Quincenal o mensual

HORARIOS DE ENTREGA:
• Lunes a viernes: 8:00-16:00 hrs
• Sábados: 8:00-12:00 hrs (previa confirmación)
• Domingos y festivos: No disponible
• Tiempo de descarga: 2-4 horas máximo

DOCUMENTACIÓN DE EMBARQUE:
• Remisión con peso y lote
• Certificado de calidad del lote
• Carta porte (complemento CFDI)
• Certificado fitosanitario (si requerido)
• Análisis de laboratorio
\end{verbatim}

\paragraph{6.2.2 Control de Calidad en
Destino}\label{control-de-calidad-en-destino}

\begin{verbatim}
PROCEDIMIENTO DE RECEPCIÓN:
1. Verificación de documentos
2. Inspección visual del producto
3. Toma de muestra para análisis (si aplica)
4. Pesaje en báscula certificada
5. Firma de acuse de recibo

CRITERIOS DE ACEPTACIÓN:
• Conformidad con especificaciones NMX
• Peso dentro de tolerancia (±2%)
• Empaque en buenas condiciones
• Documentación completa y vigente

PROCEDIMIENTO DE RECHAZO:
• Notificación inmediata (24 horas)
• Sustento técnico documentado
• Negociación de solución (reposición/descuento)
• Gastos de flete: Según causas del rechazo
\end{verbatim}

\begin{center}\rule{0.5\linewidth}{0.5pt}\end{center}

\subsection{7. SPEECH DE APROXIMACIÓN
COMERCIAL}\label{speech-de-aproximaciuxf3n-comercial}

\subsubsection{7.1 Parlamento para Contacto Telefónico
Inicial}\label{parlamento-para-contacto-telefuxf3nico-inicial}

\paragraph{7.1.1 Versión Completa (3-4
minutos)}\label{versiuxf3n-completa-3-4-minutos}

\begin{verbatim}
"Buenos días, mi nombre es Sergio Muñoz de Alba Medrano. ¿Podría comunicarme 
con el Gerente de Compras o la persona responsable de adquisición de granos básicos?

[Esperar transferencia o confirmación]

Perfecto. Le comento brevemente el motivo de mi llamada:

Represento a productores especializados de frijol negro en San Luis Potosí. 
Tenemos disponibilidad inmediata de 1,000 toneladas de frijol negro certificado 
NMX-FF-038 Grado Primera, que es una calidad superior al promedio del mercado.

He identificado que su empresa es líder en distribución de granos básicos en el 
Sureste, y considero que este producto puede ser de gran valor para su cartera 
por tres razones principales:

Primero, la calidad. Nuestro frijol negro tiene certificación Grado Primera, 
lo que significa menos del 2% de granos dañados, máximo 13% de humedad, y 
características organolépticas superiores. Esto se traduce en mejor rendimiento 
de cocción y mayor satisfacción del consumidor final.

Segundo, el origen. San Luis Potosí es reconocido nacionalmente por producir 
frijol negro de calidad premium. Nuestros productores tienen más de 15 años 
de experiencia y manejan las mejores prácticas agrícolas de la región.

Tercero, la oportunidad comercial. Nuestro precio es competitivo: $28,930 por 
tonelada, FOB su almacén. Esto les permite un margen atractivo del 20% si 
manejan precio de venta a $34,700, que sigue siendo competitivo vs. el mercado 
que está en $35,000-38,000 por tonelada.

Además, ofrecemos flexibilidad total en entregas - desde 40 toneladas hasta 
200 por embarque - y términos de pago de 30 días con 2% de descuento por 
pronto pago.

¿Le interesaría que nos reuniéramos 45 minutos para que le presente la propuesta 
completa y le entregue muestras para que su equipo técnico pueda evaluarlas?

Puedo estar en sus oficinas cualquier día de esta semana o la próxima, en el 
horario que les sea más conveniente.

[Pausa para respuesta]

Perfecto, entonces quedamos el [día] a las [hora]. Le voy a enviar un email 
ahora mismo confirmando la cita y adjuntando información preliminar para que 
puedan revisarla antes de nuestra reunión.

¿Me puede confirmar su email? [Anotar]

Excelente. Una última cosa: para optimizar el tiempo de nuestra reunión, 
¿me podría comentar qué volúmenes manejan aproximadamente de frijol negro 
al año y quiénes son sus principales proveedores actuales?

[Escuchar y tomar notas]

Perfecto, con esa información voy a preparar una propuesta específicamente 
adaptada a sus necesidades. 

Muchas gracias por su tiempo y nos vemos el [día] a las [hora]. Que tenga 
excelente día.
\end{verbatim}

\paragraph{7.1.2 Versión Ejecutiva (90
segundos)}\label{versiuxf3n-ejecutiva-90-segundos}

\begin{verbatim}
"Buenos días, habla Sergio Muñoz de Alba. ¿Es usted el responsable de compras 
de granos básicos?

Le comento rápidamente: represento a productores de San Luis Potosí con 1,000 
toneladas de frijol negro Grado Primera certificado NMX disponibles inmediatamente.

Tres puntos clave para su empresa:

Uno: Calidad superior certificada - Grado Primera NMX significa menos del 2% 
de granos dañados y mejor rendimiento de cocción.

Dos: Precio competitivo - $28,930 por tonelada FOB su almacén, que les permite 
márgenes del 20% manteniéndose competitivos en el mercado.

Tres: Flexibilidad total - Entregas desde 40 toneladas, programación según 
sus necesidades, y 30 días de crédito.

¿Podríamos reunirnos 45 minutos esta semana para que evalúe muestras y 
revise la propuesta completa?

[Pausa]

Perfecto, ¿le parece bien el [día] a las [hora]? Le confirmo por email 
con la información preliminar.

Gracias y hasta entonces.
\end{verbatim}

\subsubsection{7.2 Script para Presentación
Presencial}\label{script-para-presentaciuxf3n-presencial}

\paragraph{7.2.1 Apertura (2 minutos)}\label{apertura-2-minutos}

\begin{verbatim}
"[Nombre], muchas gracias por recibirme. Sé que su tiempo es muy valioso, 
así que voy a ser muy puntual y específico.

Como le mencioné por teléfono, mi nombre es Sergio Muñoz de Alba y represento 
a productores especializados de frijol negro en San Luis Potosí. Mi función 
es conectar productores de calidad premium con distribuidores líderes como 
ustedes que valoran la excelencia en sus proveedores.

Antes de comenzar, me gustaría conocer un poco más sobre sus necesidades 
específicas. ¿Qué volúmenes de frijol negro manejan aproximadamente al año 
y cuáles son los principales desafíos que enfrentan con sus proveedores actuales 
en términos de calidad, precio o servicio?

[Escuchar activamente y tomar notas]

Perfecto, eso me ayuda mucho a enfocar mi presentación. 

En los próximos 40 minutos les voy a mostrar cómo nuestro frijol negro San Luis 
puede no solo resolver esos desafíos, sino generar una ventaja competitiva 
real para su empresa en el mercado del Sureste.

La agenda es simple: primero, análisis del mercado y la oportunidad; segundo, 
nuestro producto y sus ventajas diferenciadas; tercero, la propuesta comercial 
específica; y cuarto, los próximos pasos para implementar.

¿Les parece bien esta estructura o tienen alguna pregunta específica que 
quieran que aborde?"
\end{verbatim}

\paragraph{7.2.2 Presentación del Producto (8
minutos)}\label{presentaciuxf3n-del-producto-8-minutos}

\begin{verbatim}
"Permítanme mostrarles el producto del que estamos hablando.

[Mostrar muestra física]

Este es nuestro frijol negro San Luis Potosí Grado Primera certificado NMX. 
Lo primero que van a notar es la uniformidad del color y el tamaño, y el 
brillo natural que tienen los granos.

Esto no es casualidad. Nuestros productores en San Luis Potosí manejan un 
proceso de producción que combina las mejores prácticas tradicionales con 
tecnología moderna:

Primero, utilizan variedades mejoradas específicamente adaptadas al terroir 
de San Luis Potosí. Este terroir - la combinación de suelo, clima y altitud - 
produce frijoles con 25% más proteína que el promedio nacional.

Segundo, el proceso de cosecha y post-cosecha es crítico. Cosechan en el punto 
óptimo de madurez y procesan inmediatamente para mantener la humedad ideal 
entre 12-13%, comparado con 15-16% que es común en el mercado.

Tercero, la selección. Cada lote pasa por un proceso de clasificación manual 
seguido de clasificación mecánica que elimina granos partidos, manchados o 
con defectos. Por eso logramos la certificación Grado Primera.

[Mostrar certificado NMX]

Esta certificación garantiza que menos del 2% de los granos tienen defectos, 
máximo 1% de impurezas, y las características organolépticas superiores que 
ustedes pueden ver.

¿Qué significa esto para ustedes en términos prácticos?

Primero, mejor experiencia del consumidor final. Este frijol se cuece en 90-120 
minutos vs. 150-180 de frijoles estándar. Tiene mejor textura, no se deshace, 
y rinde 2.3 kilos cocido por cada kilo seco.

Segundo, menos mermas y reclamaciones. La uniformidad y calidad reducen las 
devoluciones por problemas de cocción o apariencia.

Tercero, diferenciación en el mercado. Pueden posicionar este producto como 
premium y justificar un mejor precio de venta.

¿Tienen alguna pregunta sobre las características del producto?"
\end{verbatim}

\paragraph{7.2.3 Propuesta Comercial (10
minutos)}\label{propuesta-comercial-10-minutos}

\begin{verbatim}
"Ahora hablemos de números, que sé que es lo que realmente les interesa.

Basándome en lo que me comentaron sobre sus volúmenes y desafíos actuales, 
he estructurado una propuesta que les va a generar valor real:

PRECIO: Tenemos dos opciones para adaptarnos a sus necesidades:

OPCIÓN 1: $28,930 por tonelada EXW San Luis Potosí
- Ustedes se encargan del flete y seguros
- Mayor control de la logística
- Pueden negociar mejores tarifas de transporte

OPCIÓN 2: $31,441 por tonelada DDP Mérida, Yucatán
- Nosotros nos encargamos de todo hasta su almacén en Mérida
- Sin preocupaciones logísticas
- Precio todo incluido

¿Cuál prefieren? Déjenme mostrarles los números:

[Mostrar análisis en pizarrón o papel]

Si ustedes venden este producto a $34,700 por tonelada - que es 5% por debajo 
del precio promedio actual del mercado de $36,500 - obtienen un margen bruto 
del 20%, que son $5,770 por tonelada.

Pero aquí está la ventaja: al ser Grado Primera certificado, ustedes pueden 
posicionarlo como premium y venderlo hasta en $37,000-38,000 por tonelada, 
lo que les daría márgenes del 25-30%.

Sus clientes van a pagar este premium porque:
- Van a tener menos quejas de consumidores
- Mejor rotación por la calidad superior
- Diferenciación vs. competencia que maneja Segunda y estándar

VOLÚMENES: Tenemos 1,000 toneladas disponibles inmediatamente. Basándome en 
lo que me comentaron, sugiero que empecemos con 100 toneladas para validar 
el mercado, y luego programemos 150-200 toneladas mensuales.

TÉRMINOS: 30 días de crédito desde factura, con 2% de descuento si pagan a 
15 días. Entregas mínimas de 40 toneladas, máximas de 200, programadas con 
una semana de anticipación.

DESCUENTOS POR VOLUMEN: Si se comprometen a 500 toneladas anuales, el precio 
baja a $28,062 por tonelada, que mejora aún más sus márgenes.

¿Qué les parece esta estructura? ¿Tienen alguna pregunta sobre los números?"
\end{verbatim}

\paragraph{7.2.4 Manejo de Objeciones
Comunes}\label{manejo-de-objeciones-comunes}

\begin{verbatim}
OBJECIÓN: "El precio está alto comparado con nuestros proveedores actuales"

RESPUESTA: 
"Entiendo perfectamente esa preocupación, es natural comparar precios. 
Permítame explicarle por qué este precio realmente les va a generar más 
utilidad neta:

Primero, comparemos manzanas con manzanas. Nuestro Grado Primera vs. su 
proveedor actual que probablemente es Segunda. La diferencia de calidad 
les permite vender 8-12% más caro, lo que más que compensa el diferencial 
de compra.

Segundo, menos mermas. Con Grado Primera van a tener máximo 2% de reclamaciones 
vs. 5-8% que es típico con calidades inferiores. Eso son $1,500-3,000 por 
tonelada de ahorro.

Tercero, mejor rotación. La calidad superior significa que sus clientes van 
a recomprar más rápido, mejorando su flujo de caja.

Si hacemos la cuenta completa - mejor precio de venta, menos mermas, mejor 
rotación - su utilidad neta va a ser 15-20% superior. ¿Le parece si hacemos 
una prueba con 40 toneladas para que comprueben estos números?"

OBJECIÓN: "Ya tenemos proveedores establecidos"

RESPUESTA:
"Perfecto, eso habla muy bien de ustedes - mantener relaciones de largo plazo 
es clave en este negocio. No estoy sugiriendo que cambien proveedores, sino 
que agreguen una opción premium a su portafolio.

Piénsenlo así: seguramente tienen clientes que buscan calidad superior y están 
dispuestos a pagar por ella. Este producto les permite atender ese segmento 
sin afectar su negocio base.

Además, tener múltiples proveedores les da poder de negociación y reduce riesgos 
de desabasto. ¿No creen que diversificar con una opción de calidad superior 
les daría más flexibilidad comercial?"

OBJECIÓN: "Necesitamos evaluar con nuestro equipo técnico"

RESPUESTA:
"Por supuesto, es exactamente lo que yo haría en su lugar. Por eso traje estas 
muestras y toda la documentación técnica.

Sugiero que su equipo haga las pruebas de cocción y compare con sus productos 
actuales. Van a ver inmediatamente la diferencia en tiempo de cocción, textura 
y rendimiento.

¿Qué les parece si les doy 10 días para la evaluación técnica, y mientras 
tanto yo preparo una propuesta más detallada basada en lo que hemos platicado 
hoy? Podríamos reunirnos nuevamente la próxima semana para conocer sus resultados 
y afinar los detalles."
\end{verbatim}

\paragraph{7.2.5 Cierre y Próximos Pasos (5
minutos)}\label{cierre-y-pruxf3ximos-pasos-5-minutos}

\begin{verbatim}
"Perfecto, creo que hemos cubierto los puntos principales. 

Para resumir lo que hemos acordado:

Les voy a dejar estas muestras y documentación para que su equipo técnico 
haga las evaluaciones correspondientes. El análisis de laboratorio que incluyo 
es del mismo lote de la muestra, así que van a poder verificar que los números 
coincidan con sus propias pruebas.

Ustedes van a hacer las pruebas de cocción y evaluación organoléptica en los 
próximos 7 días, y me van a dar feedback el [fecha específica].

Mientras tanto, yo voy a preparar una propuesta formal más detallada, incluyendo 
un cronograma de entregas sugerido basado en sus volúmenes actuales y un análisis 
de rentabilidad específico para su operación.

Nos volvemos a reunir el [fecha] para conocer sus resultados y, si todo está 
bien, definir los términos finales del contrato.

¿Hay algo más que necesiten para tomar la decisión? ¿Alguna información adicional 
o algún análisis específico?

[Esperar respuestas y tomar notas]

Perfecto. Una pregunta final: si los resultados de las pruebas son positivos 
y llegamos a un acuerdo, ¿cuándo estarían listos para recibir la primera entrega? 
Esto me ayuda a coordinar la logística con anticipación.

[Anotar timeline]

Excelente. Entonces nos vemos el [fecha] a las [hora]. Mientras tanto, cualquier 
pregunta que surja durante las evaluaciones, no duden en llamarme directamente 
a este número [entregar tarjeta].

Muchas gracias por su tiempo y por la oportunidad de presentarles nuestra propuesta. 
Estoy muy optimista de que vamos a poder trabajar juntos y generar una relación 
comercial muy exitosa.

Que tengan excelente tarde."
\end{verbatim}

\begin{center}\rule{0.5\linewidth}{0.5pt}\end{center}

\subsection{ANEXOS}\label{anexos}

\subsubsection{Anexo A: Checklist de
Preparación}\label{anexo-a-checklist-de-preparaciuxf3n}

\begin{verbatim}
DOCUMENTOS TÉCNICOS:
□ Certificado NMX-FF-038 vigente
□ Análisis de laboratorio (<15 días)
□ Certificado fitosanitario SENASICA
□ Ficha técnica completa
□ Fotografías profesionales del producto

DOCUMENTOS COMERCIALES:
□ RFC y cédula fiscal
□ Estados financieros básicos
□ Referencias comerciales (mínimo 3)
□ Póliza de seguro responsabilidad civil
□ Propuesta comercial formal

MATERIALES DE PRESENTACIÓN:
□ Presentación PowerPoint actualizada
□ Muestras físicas (10 kg empacadas)
□ Kit de evaluación técnica
□ Calculadora y material de apoyo
□ Tarjetas de presentación
\end{verbatim}

\subsubsection{Anexo B: Contactos Clave del
Sector}\label{anexo-b-contactos-clave-del-sector}

\begin{verbatim}
ORGANISMOS REGULADORES:
• SENASICA: 01 800 751 2100
• COFEPRIS: 01 800 033 5050
• EMA (Entidad Mexicana Acreditación): 55 5663 9300

LABORATORIOS CERTIFICADOS:
• SGS México: 55 5387 1717
• Bureau Veritas: 55 5262 1501
• CIATEJ: 33 3345 5200

FUENTES DE INFORMACIÓN:
• SIAP (precios): www.gob.mx/siap
• SNIIM (mercados): www.economia-sniim.gob.mx
• Bolsa de Cereales: www.bolsadecereales.com
\end{verbatim}

\subsubsection{Anexo C: Calendario de
Temporadas}\label{anexo-c-calendario-de-temporadas}

\begin{verbatim}
TEMPORADAS DE FRIJOL EN MÉXICO:

PRIMAVERA-VERANO (Abril-Agosto):
• Siembra: Abril-Mayo
• Cosecha: Julio-Agosto
• Comercialización: Agosto-Febrero
• Volúmenes: 70% de la producción nacional

OTOÑO-INVIERNO (Octubre-Febrero):
• Siembra: Octubre-Noviembre
• Cosecha: Enero-Febrero
• Comercialización: Febrero-Junio
• Volúmenes: 30% de la producción nacional

PICOS DE DEMANDA:
• Enero-Marzo: Post-fiestas decembrinas
• Agosto-Octubre: Regreso a clases
• Noviembre-Diciembre: Fiestas navideñas
\end{verbatim}

\begin{center}\rule{0.5\linewidth}{0.5pt}\end{center}

\textbf{DOCUMENTO CONFIDENCIAL - USO EXCLUSIVO PARA COMERCIALIZACIÓN DE
FRIJOL NEGRO SAN LUIS POTOSÍ}

\textbf{Elaborado por:} Sergio Muñoz de Alba Medrano\\
\textbf{Fecha:} Diciembre 2025\\
\textbf{Versión:} 1.0
